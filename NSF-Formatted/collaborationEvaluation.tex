\documentclass{article}
\usepackage[utf8]{inputenc}
\usepackage{amsmath}
\usepackage{amssymb}
\usepackage{fullpage}
\usepackage{color}

\usepackage{textcomp}
\usepackage{ifthen}
\usepackage{listings}
\usepackage{fancyvrb}
\usepackage{verbatim}
%\usepackage[noend]{algorithm2e}

%\usepackage[pdftex]{graphicx}
\usepackage{graphicx}
\usepackage{subfig}
\usepackage{calc}
\usepackage{float}

\usepackage{url}

% For generating custom lists (similar to table of contents, list of figures
% etc.)
\usepackage[subfigure]{tocloft}

\usepackage{hyperref}
\usepackage[all]{hypcap}
\usepackage{latexsym}

\usepackage{enumitem}

% Remove spacing around section headings
\usepackage[compact]{titlesec}
\titlespacing{\section}{0pt}{*0}{*0}
\titlespacing{\subsection}{0pt}{*0}{*0}
\titlespacing{\subsubsection}{0pt}{*0}{*0}

%\usepackage[left=1.05in,right=1.05in,top=1in,bottom=1in]{geometry}

% tweak or comment out the baselinestretch to control how much space for hand-writing comments you get between the lines
%\renewcommand{\baselinestretch}{1.8}

% Compress whitespace around itemize and enumerate
\setitemize{topsep=.0em,parsep=0pt,partopsep=0pt,labelsep=0.2em,itemsep=0pt,leftmargin=0.5em}
\setenumerate{topsep=.0em,parsep=0pt,partopsep=0pt,labelsep=0.2em,itemsep=0pt,leftmargin=0.5em}

% Remove paragraph indentation
\setlength{\parindent}{0pt}

% Remove paragraph spacing
\setlength{\parskip}{0pt}
\setlength{\parsep}{0pt}
\setlength{\headsep}{0pt}
\setlength{\topskip}{0pt}
\setlength{\topmargin}{0pt}
\setlength{\topsep}{0pt}
\setlength{\partopsep}{0pt}

% Decrease the line spread slightly
\linespread{1.02}


\DefineVerbatimEnvironment%
{tree}{Verbatim}{frame=single,fontsize=\scriptsize}

\DefineVerbatimEnvironment%
{treeNumber}{Verbatim}{frame=single,numbers=left,fontsize=\scriptsize}
\title{Collaboration and Evaluation} 

\newtheorem{definition}{Definition}

%\newcommand{\comment}[3]{}  % suppress comments
\renewcommand{\comment}[3]{{\color{#1} {\bf #2 :} #3}}

\newcommand{\yoav}[1]{\comment{magenta}{Yoav}{#1}}
\newcommand{\piya}[1]{\comment{blue}{Piya}{#1}}
\newcommand{\peter}[1]{\comment{pink}{Peter}{#1}}
\newcommand{\rayan}[1]{\comment{red}{Rayan}{#1}}
\newcommand{\alex}[1]{\comment{green}{Alex}{#1}}

%math symbols
\newcommand{\R}{\mathbb{R}}
\newcommand{\rad}{\mathcal{R}}
\newcommand{\sign}{\mathrm{sign}}

\newcommand{\state}{\theta}
\newcommand{\estate}{\hat{\theta}}
\newcommand{\Vstate}{\Theta}
\newcommand{\eVstate}{\hat{\Theta}}
\newcommand{\Sstate}{\psi}
\newcommand{\eSstate}{\hat{\psi}}
\newcommand{\VSstate}{\Psi}
\newcommand{\eVSstate}{\hat{\Psi}}
\newcommand{\env}{\mathbf{E}}
\newcommand{\transfer}{\mathbf{\Phi}}

\newcommand{\signal}{\mathbf{x}}
\newcommand{\Vsignal}{\mathbf{X}}
\newcommand{\dsignal}{\mathbf{y}}
\newcommand{\Vdsignal}{\mathbf{Y}}
\newcommand{\dtime}{t}
\newcommand{\ctime}{\tau}


\begin{document}
Our team consists of University of California San Diego (UCSD) faculty members from the departments of Computer Science (Yoav Freund), Electrical Engineering (Piya Pal, Peter Gerstoft), and Mathematics (Alex Cloninger, Rayan Saab), directly representing three of the four communities of HDR TRIPODS.  Additionally, all members of the team are founding members of UCSD's Halicioglu Data Science Institute (HDSI), a new academic unit at UCSD with a focused mission. Its mission, one that it shares to a large extent with HDR TRIPODS, is to lay the groundwork for the scientific foundations of this emerging discipline, develop new methods and infrastructure, and train students, faculty and industrial partners to use data science in ways that will allow them to solve some of the world’s most pressing problems \cite{the HDSI website}. 

Having all our team members located on the same campus and regularly interacting through HDSI will greatly enhance our ability to collaborate closely on the proposed projects. In addition to the direct and frequent interaction of the faculty members involved in this proposal, with the NSF's support, we intend to jointly supervise PhD students and to co-mentor a postdoctoral fellow. 

Additionally, we note that our team includes theoreticians who also have a deep interest in and knowledge of applications, and all  have worked closely and published papers with practitioners. In the context of this proposal, Co-PI Gerstoft's intimate knowledge of sensors and sensor networks will allow us to account for practical issues that arise in data science, including (among others) its multi-modality and incompleteness. Our awareness of practical issues will hopefully allow us to maximize the impact of the algorithms and methodologies that result from our work.  

In what follows, we will describe our expertise, plans for advising joint students, for collaborating, as well as for evaluating the results of our work.  
 
\section{Expertise}

%The members of our team are: Computer Science: Yoav Freund,
%Electrical engineering: Piya Pal and Peter Gerstoft.
%Mathematics: Rayan Saab and Alex Cloninger.

\begin{itemize}
\item {\bf PI Freund's} expertise is in Computational Learning Theory,
  and statistics. Among his theoretical work are Boosting~\cite{},
  statistical analysis of the generalization error for ensamble
  classifiers~\cite{}, online learning~\cite{}, learning and game
  theory~\cite{} unsupervised learning algorithm~\cite{}.  In
  addition, PI Freund has worked on applications of machine learning
  to image analysis~{} and, in particular, image analysis for
  biological microscopy~\cite{}.
\item {\bf Co-PI Cloninger's} expertise is in applied harmonic
  analysis and the analysis of high dimensional data.  He focuses on
  approaches that model the data as being locally lower dimensional,
  including data concentrated near manifolds or subspaces.  These
  types of problems arise in a number of scientific disciplines,
  including imaging, medicine, and artificial intelligence, and the
  techniques developed relate to a number of machine learning and
  statistical algorithms, including deep learning, network analysis,
  and measuring distances between probability distributions.
  % Alex has worked with a number of collaborators in scientific disciplines outside math on data intensive problems, including collaborations with researchers at National Institutes of Health and National Geospatial Administration, and economics , physics, and medical departments at various universities.

\item {\bf Co-PI Gerstoft's} focus on data-driven computational geophysics and within these fields I further cover the subtopics: applied signal processing, inverse methods, mathematical models, extracting information from noise, and machine learning. I currently focus on developing new sensing techniques using large array sensor data. Data science methods (big data/machine learning) and compressive sensing is a main focus for sensing the physical environment.  These techniques are applied to observing tsunamis, earthquakes, traffic, Antarctic signals, as well as extracting environmental information from just noise. 
\item {\bf Co-PI Pal's} expertise... \yoav{please add yourself}
\item {\bf Co-PI Saab's} expertise is in applied and computational harmonic analysis, and in mathematical signal processing. He is interested in, and has published extensively on questions regarding efficient acquisition, quantization, representation, and processing of data. He has studied these problems in classical contexts, such as those of band limited functions, but also in modern contexts, such as compressed sensing of structured signals, be they sparse vectors, low-rank matrices, or signals from arbitrary sets. He has also worked on signal and data processing problems related to multiple sensors, including the blind source separation problem. In his work, Saab uses and develops a variety of tools from high dimensional probability theory, applied harmonic analysis, convex analysis and optimization, and quantization theory, among others. 
\end{itemize}

\section{Collaboration Plan}
%\iffalse
{\bf Joint Students and Post Doc}\\
 In
addition to its focus on the proposed scientific content, a main goal
of this proposal is to recruit and train highly qualified personnel
for the workforce of tomorrow. Exploiting the fact that our team
members are all UCSD faculty with affiliations to HDSI, we will
recruit and advise four graduate students and one post-graduate student as a group. We
will actively recruit students who have a strong mathematical
background and are able and willing to implement algorithms as
reusable code. Equally importantly, we will actively seek out students
from underrepresented groups in STEM.

Each student will have two advisors, from two of the three
disciplines, who will work closely with the student. From a training
point of view, this will help the student gain a broader perspective
on data science as a field while acquiring an in-depth understanding
of tools and methods from more than one area. From a collaboration
point of view, joint advising will provide one of the mechanisms by
which our team members will work together.

The Post Doc will assist PI Freund in keeping track of all activities and work with each of the four PIs on a per need base as well as connected with some of the graduate students. We will make sure that post Doc is actively involved in the project.

{\bf Collaboration}\\  Each
graduate student will have a one hour meeting with each of their their
advisors each week. The student and both advisors will also have
regular bi-weekly meetings to assess progress and ensure continuing
synergy on each project.  In addition, we will have a one and a half
hour weekly group meeting, where each meeting will include all PIs and
students, and where one of the students or PIs will present to the
others.

%\rayan{not sure about this one, get rid of it if you don't like it}
%Finally, we will have annual retreats where all the team members meet in order to discuss our progress and plan our work in the coming year.
%\fi

%\yoav{other version}\\
{\bf Joint Students}\\
%\rayan{Shouldn't we mention the postdoc here or elsewhere?}
In addition to its focus on the proposed scientific content, a main goal of this proposal is to recruit and train highly qualified personnel for the workforce of tomorrow. Exploiting the fact that our team members are all UCSD faculty with affiliations to HDSI, we will recruit and advise graduate and post-graduate students as a
group. We will actively recruit students who have a strong
mathematical background and are able and willing to implement
algorithms as reusable code. Equally importantly, we will actively seek out students from underrepresented groups in STEM. 

%Each student will have two advisors, from two of the three disciplines, who will work closely with the student. From a training point of view, this will help the student gain a broader perspective on data science as a field while acquiring an in-depth understanding of tools and methods from more than one area. From a collaboration point of view, joint advising will provide one of the mechanisms by which our team members will work together. 

{\bf Collaboration}\\ %\rayan{see if you like these edits.}
The graduate students will glue our team together and each graduate student will have at least two PI-advisers.  
Each graduate student will have a one hour meeting with their
two advisors each week. The student and both advisors will also have regular bi-weekly meetings to assess progress and ensure continuing synergy on each project.  In addition, we will have a weekly group meeting with all PIs and students, and where one of the students or PIs will present their work. 

%\rayan{not sure about this one, get rid of it if you don't like it.}
%\peter{\bf OK. Because it is written here does not mean we do it.}
Finally, we will have a kick-off retreat and followed by annual retreats where all the team members meet to assess our progress on the proposed work.

{\bf Evaluation}\\
\rayan{not sure what they want here?}
We plan to submit paper to the following conferences and journals:
Science, Nature, COLT, ICML, AIstat, JMLR, IEEE Signal Processing, IEEE Information
theory, Applied and Computational Harmonic Analysis, Foundations of Computational Mathematics, Information and Inference, SIAM Journal on Mathematics of Data Science, please add.

We Will implement our algorithms in python and jupyter notebooks and
make them available through GitHUB.

We will use several benchmarks and dataset to evaluate the accuracy of the developed approaches:\\
(1) We intend to compare our performance with the DCASE challenge: Two tasks in acoustics and audio signal processing that have befitted from deep leaning are sound event detection and source localization. These methods replace physics-based acoustic propagation models or hand-engineered detectors with deep-learning architectures in the 2017 Detection and Classification of Acoustic Scenes and Events (DCASE) challenge.\cite{mesaros2017dcase}
This is an annual event, so new datasets will be available.
\\
(2) For seismic data it is now common to use massive sensor data. We have access to the  5000 geophones collected in a 7x10 km area Long Beach over months at a sampling rates of 500 Hz. We have worked on the data in e.g.\ \cite{riahi2017}. many other data sets are available at the Iris website, \href{iris.edu}.
(3)
We will collect audio data in Prof.\ Christiansen's instrumented living
quarter and evaluate out algorithms on the collected data.


We will evaluate our performance in terms of:
\begin{enumerate}
\item Number of publications and quality of publication venues.
\item Number of (open source) implementations.
\item Performance of our methods on benchmark data.
\end{enumerate}

{\bf Institute for Learning Networks}\\
\yoav{Please suggest names for the institute}
\peter{Institute for  Sensor Network Data Science}

We plan to join forces with researchers in the areas of embedded
computers, robotics and medical sensing.

All PIs are members of the Haligliou Data Science Institute (HDSI).

\bigskip

\rayan{Below are ``Additional Solicitation Specific Review Criteria" from the call that I attempted to address above - need help on this!:}
\begin{itemize}
    \item Does the proposal describe a well-integrated research and training program focused on the theoretical foundations of data science and fostering collaboration and interaction among the targeted (at least three of the four) communities of HDR TRIPODS?
    \rayan{not sure about the well-integrated, but I think we hit the second point}
    \item Does the proposal address the “broad themes of the program” listed in the Program Description?
    \rayan{I tried to hit a few of these points here, but we NEED to go back and make edits to the proposal to make sure we have decent coverage.}
    \item Does the proposal address strategies for workforce development, including but not limited to novel educational and training activities and efforts toward full participation of groups underrepresented in science, technology, engineering and mathematics (STEM)? \rayan{We should make explicit mention of these things in the Broader Impacts section. For example, I am currently advising two undergrads on a project with industry (through HDSI). We need to capitalize on such things more, particularly in the context of underrepresented groups. }
    
    \item Transdisciplinarity/Synergy: Is the project transdisciplinary, bringing together theories and approaches from at least three of electrical engineering, mathematics, statistics, and theoretical computer science? Is there synergy between the different groups?\rayan{I think this is fine}
    
    \item Vision: Is there a strong case for the ability to identify and articulate a vision for the foundations of data science?
    \rayan{good question}
    
    \item Quality and Value of Collaboration: Is the expertise of the PIs complementary and well-suited to the research and training programs developed in this project? Are the specific roles of each collaborating investigator clear? Does the collective team have expertise representing at least three of the four communities (electrical engineering, mathematics, statistics, and theoretical computer science)?\rayan{fine?}
    
    \item Is there a well-developed plan for collaboration and interaction with the domain sciences and industry? \rayan{help?}
    
    \item Does the proposal provide a clear plan and rationale for an investment of the size proposed, including clear plans to develop capacity for potential future Phase II operations? \rayan{?}
    
    \item Does the Collaboration and Evaluation Plan identify clear measures of success, both for Phase I operations and development of capability for a potential Phase II, along with a plan to evaluate the project with respect to those measures by gathering quantitative and qualitative data? \rayan{?}
    
    \item Does the Collaboration and Evaluation Plan provide a clear plan for thoughtful, ongoing assessment of all Institute activities? How will the assessment be used to inform and improve both daily Institute operations and long-range planning, aiming toward a successful Phase II Institute? \rayan{?}

\end{itemize}



{
\bibliographystyle{abbrv}
%\bibliography{plato,tripods}
\bibliography{tripods}
}

\end{document}

% Collaboration and Evaluation Plan as a separate Supplementary Document
% (limited to 5 pages). This plan must describe the expertise in the
% relevant disciplines provided by the PIs as required above under "Who
% May Serve as PI" as well as plans for working together to meet the
% goals of the program. The Collaboration and Evaluation Plan must also
% describe clear measures of success for the project, including
% developing capability and capacity for a potential Phase II, and a
% plan for evaluating success. Proposals without this document will be
% returned without review.


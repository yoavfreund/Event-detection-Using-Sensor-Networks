\documentclass{article}
\usepackage[utf8]{inputenc}
\usepackage{amsmath}
\usepackage{amssymb}
\usepackage{fullpage}
\usepackage{color}

\usepackage{textcomp}
\usepackage{ifthen}
\usepackage{listings}
\usepackage{fancyvrb}
\usepackage{verbatim}
%\usepackage[noend]{algorithm2e}

%\usepackage[pdftex]{graphicx}
\usepackage{graphicx}
\usepackage{subfig}
\usepackage{calc}
\usepackage{float}

\usepackage{url}

% For generating custom lists (similar to table of contents, list of figures
% etc.)
\usepackage[subfigure]{tocloft}

\usepackage{hyperref}
\usepackage[all]{hypcap}
\usepackage{latexsym}

\usepackage{enumitem}

% Remove spacing around section headings
\usepackage[compact]{titlesec}
\titlespacing{\section}{0pt}{*0}{*0}
\titlespacing{\subsection}{0pt}{*0}{*0}
\titlespacing{\subsubsection}{0pt}{*0}{*0}

%\usepackage[left=1.05in,right=1.05in,top=1in,bottom=1in]{geometry}

% tweak or comment out the baselinestretch to control how much space for hand-writing comments you get between the lines
%\renewcommand{\baselinestretch}{1.8}

% Compress whitespace around itemize and enumerate
\setitemize{topsep=.0em,parsep=0pt,partopsep=0pt,labelsep=0.2em,itemsep=0pt,leftmargin=0.5em}
\setenumerate{topsep=.0em,parsep=0pt,partopsep=0pt,labelsep=0.2em,itemsep=0pt,leftmargin=0.5em}

% Remove paragraph indentation
\setlength{\parindent}{0pt}

% Remove paragraph spacing
\setlength{\parskip}{0pt}
\setlength{\parsep}{0pt}
\setlength{\headsep}{0pt}
\setlength{\topskip}{0pt}
\setlength{\topmargin}{0pt}
\setlength{\topsep}{0pt}
\setlength{\partopsep}{0pt}

% Decrease the line spread slightly
\linespread{1.02}


\DefineVerbatimEnvironment%
{tree}{Verbatim}{frame=single,fontsize=\scriptsize}

\DefineVerbatimEnvironment%
{treeNumber}{Verbatim}{frame=single,numbers=left,fontsize=\scriptsize}
\title{Collaboration and Evaluation}

\newtheorem{definition}{Definition}

%\newcommand{\comment}[3]{}  % suppress comments
\renewcommand{\comment}[3]{{\color{#1} {\bf #2 :} #3}}

\newcommand{\yoav}[1]{\comment{magenta}{Yoav}{#1}}
\newcommand{\piya}[1]{\comment{blue}{Piya}{#1}}
\newcommand{\peter}[1]{\comment{pink}{Peter}{#1}}
\newcommand{\rayan}[1]{\comment{red}{Rayan}{#1}}
\newcommand{\alex}[1]{\comment{green}{Alex}{#1}}

%math symbols
\newcommand{\R}{\mathbb{R}}
\newcommand{\rad}{\mathcal{R}}
\newcommand{\sign}{\mathrm{sign}}

\newcommand{\state}{\theta}
\newcommand{\estate}{\hat{\theta}}
\newcommand{\Vstate}{\Theta}
\newcommand{\eVstate}{\hat{\Theta}}
\newcommand{\Sstate}{\psi}
\newcommand{\eSstate}{\hat{\psi}}
\newcommand{\VSstate}{\Psi}
\newcommand{\eVSstate}{\hat{\Psi}}
\newcommand{\env}{\mathbf{E}}
\newcommand{\transfer}{\mathbf{\Phi}}

\newcommand{\signal}{\mathbf{x}}
\newcommand{\Vsignal}{\mathbf{X}}
\newcommand{\dsignal}{\mathbf{y}}
\newcommand{\Vdsignal}{\mathbf{Y}}
\newcommand{\dtime}{t}
\newcommand{\ctime}{\tau}


\begin{document}
\section{Collaboration and Evaluation Plan}
The members of our team are: Computer Science: Yoav Freund,
Electrical engineering: Piya Pal and Peter Gerstoft.
Mathematics: Rayan Saab and Alex Cloninger.

\begin{itemize}
\item {\bf PI Freund's} expertise is in Computational Learning Theory and
Computer Science. Among his theoretical work are 
Boosting~\cite{}, online learning~\cite{}, learning and game
theory~\cite{} and the RP-tree algorithm~\cite{}. He has also worked
on applications of machine learning to image analysis for
biology~\cite{}.
\item please add yourself.
\end{itemize}

Description of how we shall colaborate

{\bf Joint Students}\\
We will recruit and advise graduate and post-graduate students as a
group. We will actively recruit students who have a strong
mathematical background and are able and willing to implement
algorithms as reusable code.  Each student will have two advisors,
from two of the three disciplines.

{\bf Collaboration}\\
Each graduate student will have a one hour meeting with each other
their advisors each week.

We will have a one and a half hour weekly group meeting, which will
include all PIs and students, in each of which one of the students or
PIs will present to the others.

{\bf Evaluation}\\
We plan to submit paper to the following conferences and journals:
COLT, ICML, AIstat, JMLR, IEEE Signal Processing, IEEE Information
theory, please add.

We Will implement our algorithms in python and jupyter notebooks and
make them available through GitHUB.

We will use several benchmarks to evaluate the accuracy.
\yoav{Piya, Peter, can you suggest benchmark data sets?}\\
We will collect audio data in Prof. Christiansen's intrumented living
quarter and evaluate out algorithms on the collected data.

We will evaluate our performance in terms of:
\begin{enumerate}
\item number of publications.
\item number of implementations.
\item Performance of our methods on benchmark data.
\end{enumerate}

{\bf Institute for Learning Sensor Networks}\\
\yoav{Please suggest names for the institute}

In phase II we plan to 

We plan to join forces with researchers in the areas of embedded
computers, robotics and medical sensing.

All PIs are members of the Haligliou Data Science Institute (HDSI).

{
\bibliographystyle{abbrv}
%\bibliography{plato,tripods}
\bibliography{tripods}
}

\end{document}

% Collaboration and Evaluation Plan as a separate Supplementary Document
% (limited to 5 pages). This plan must describe the expertise in the
% relevant disciplines provided by the PIs as required above under "Who
% May Serve as PI" as well as plans for working together to meet the
% goals of the program. The Collaboration and Evaluation Plan must also
% describe clear measures of success for the project, including
% developing capability and capacity for a potential Phase II, and a
% plan for evaluating success. Proposals without this document will be
% returned without review.



\section{results from previous grants}
\subsection{Sampling and quantization theorems for modern data acquisition}
%The investigator has not received prior NSF support.
%\vspace{5pt}\noindent{\bf Results from Prior NSF Support:} 
{Co-PI Saab: \em Sampling and quantization theorems for modern data acquisition (DMS-1517204, 08/01/2015--07/31/19,
\$160,404)} 
%   So far this grant has resulted in three journal articles \cite{KSW16, MS16, NSW16} and one journal submission \cite{SWY16}. \\
%   {\bf Intellectual Merit:} Two of the above works \cite{KSW16, SWY16} pertain to the quantization of compressed sensing measurements. Specifically, \cite{KSW16} is the first result showing that \emph{scalar} 1-bit quantization allows the magnitude of sparse signals to be recovered from compressed sensing measurements, provided an affine shift is employed. \cite{SWY16} shows that a combination of Sigma-Delta quantization and an encoding step based on a discrete Johnson-Lindenstrauss embedding allows for a near-optimal rate-distortion relationship. The other two papers quantify how using prior information (in the form of a possibly erroneous support estimate) in conjunction with weighted $\ell_1$ minimization allows for improved recovery guarantees from fewer compressed sensing measurements when the support estimate is accurate enough. 
% \\ {\bf Broader Impacts: }  In addition to dissemination of research results through seminars and workshop talks, Saab has developed and taught graduate courses on (1) compressed sensing and its applications and (2) applied and computational harmonic analysis. Additionally, he has mentored a (former) undergraduate (Steven Gagniere, currently at UCLA) in research on quantization of compressed sensing measurements (paper currently being written).   
This grant resulted in 9 published or accepted journal articles \cite{KSW16, MS16, NSW16, SWY16, NSW17, LS2018, IPSV2018, HS2018, FKS17}, and 4 conference papers \cite{NSW17_conf, FKS17_conf, IPSV17_conf, ILNS19}.  {\bf Intellectual Merit:}  \cite{KSW16, SWY16, FKS17, LS2018, ILNS19} study quantization of compressed sensing (CS) measurements,  dealing with  $1$-bit scalar quantization, compression of  bit-streams,  circulant measurement matrices, and low-rank matrices, and manifold-valued signals. \cite{HS2018} develops a state-of-the-art technique, and accompanying theory, for embedding datasets into the binary cube while preserving Euclidean distances.  \cite{NSW17} provides a framework for using $1$-bit  measurements for classification  and analyzes the case of two clusters. The papers \cite{MS16, NSW16} show that using prior support information and weighted $\ell_1$ minimization yield improved recovery  from fewer CS measurements. Finally, \cite{IPSV2018, IPSV17_conf} study phase retrieval from local measurements. The works use and develop tools in random matrix theory, mathematical signal processing, and applied harmonic analysis.  
  %Specifically, \cite{KSW16} is the first result showing that \emph{scalar} 1-bit quantization allows magnitude recovery from compressed sensing measurements, provided an affine shift is employed. \cite{SWY16} shows that Sigma-Delta quantization with an encoding step based on a discrete Johnson-Lindenstrauss embedding allows for a near-optimal rate-distortion relationship. \cite{FKS}The other two papers quantify how using prior information (in the form of a possibly erroneous support estimate) in conjunction with weighted $\ell_1$ minimization allows for improved recovery guarantees from fewer compressed sensing measurements when the support estimate is accurate enough. 
 {\bf Broader Impacts: }  In addition to dissemination of results through multiple seminars and workshop talks, Saab has developed and taught graduate courses on (1) compressed sensing and its applications, (2) applied and computational harmonic analysis, and (3) mathematical methods in data science. He has  mentored a UCSD undergraduate (S. Gagniere, currently at UCLA) in research on quantization of CS measurements, and UCSD graduate students (E. Lybrand, A. Nelson, B. Preskitt, J. Liang), and postdocs (A. Ma, T. Huynh),  some of whom are co-authors on the works above, while others are taking part in ongoing work. Two of the above mentioned graduate students and postdocs are female, one is an air-force officer, and at least two are members of under-represented groups.%\hfill\\

{\bf  P. Gerstoft}: 2014-2017; PLR-1246151 (SIO; 844k); Collaborative Research: Dynamic Response of the Ross Ice Shelf to Wave-induced Vibrations. Intellectual Merit: This project (DRIS) investigates the response of ice shelves to ocean wave forcing to (1) infer bulk elastic properties from signal propagation characteristics, and (2) to determine how IG wave and other gravity wave forcing propagates across the Ross Ice Shelf (RIS), and (3) to monitor seasonal variability of the RIS response and icequake activity. In 2014, 16 DRIS broadband stations were installed to acquire data for 2 years. Subsequently, 13 geodetic GPS stations were installed during the 2015 field season recovery of the first year’s seismic data. Joint processing of the 2016 seismic and GPS data are underway. Broader Impacts: Supported 2 postdocs, 2 graduate students. NSF Artists and Writers awardee Glenn McClure produced symphonic and choral works from the DRIS data. The Birch Aquarium at Scripps is developing an Antarctic exhibit featuring DRIS results. Publications: [Bromirski et al., 2015; Diez et al., 2015; Bromirski et al., 2017; Chen et al., 2018]. A project web site [Bromirski, 2014], suitable for informing the public, is maintained. 

\section{Results from previous grants}
%\subsection{Sampling and quantization theorems for modern data acquisition}
%The investigator has not received prior NSF support.
%\vspace{5pt}\noindent{\bf Results from Prior NSF Support:} 
{\bf R. Saab:} {\em Sampling and quantization theorems for modern data acquisition (DMS-1517204, 08/01/2015--07/31/19,
\$160,404)}
This grant resulted in 9 published or accepted journal articles \cite{knudson2016one, mansour2017recovery, needell2017weighted, SaabIEEEIT, needell2018simple, LybrandSaab2018, iwen2018phase, huynh2018fast, feng2019quantized}, and 4 conference papers \cite{needell2017simple, feng2017quantized, iwen2018phase, iwen2019new}.  {\bf Intellectual Merit:}  \cite{knudson2016one, SaabIEEEIT, feng2019quantized, LybrandSaab2018, iwen2019new} study quantization of compressed sensing (CS) measurements, under various signal, measurement, and quantization models. \cite{huynh2018fast} develops techniques and theory for embedding data into the binary cube while  \cite{needell2018simple} provides a framework for using $1$-bit  measurements for classification. \cite{mansour2017recovery, needell2017weighted} study  the use of prior support information  in CS. \cite{iwen2018phase, iwen2017phase} study phase retrieval. %The works use and develop tools in random matrix theory, mathematical signal processing, and applied harmonic analysis. 
{\bf Broader Impacts: }  Saab disseminated the results through multiple invited talks, and has developed and taught graduate courses on (1) compressed sensing and its applications, (2) applied and computational harmonic analysis, and (3) mathematical methods in data science. He mentored a UCSD undergraduate (currently a PhD student at UCLA), and advised four UCSD graduate students and two postdocs,  some of whom are co-authors on the works above, while others are taking part in ongoing work. Two of the  mentees are female, and two are members of under-represented groups.%\hfill\\

{\bf  P. Gerstoft}: 2014--2018; PLR-1246151 (SIO; \$844k); Collaborative Research: Dynamic Response of the Ross Ice Shelf to Wave-induced Vibrations. {\bf Intellectual Merit}: This project (DRIS) investigates the response of ice shelves to ocean wave forcing to (1) infer bulk elastic properties from signal propagation characteristics, and (2) to determine how IG wave and other gravity wave forcing propagates across the Ross Ice Shelf (RIS), and (3) to monitor seasonal variability of the RIS response and icequake activity. In 2014, 16 DRIS broadband stations were installed to acquire data for 2 years. Subsequently, 13 geodetic GPS stations were installed during the 2015 field season recovery of the first year’s seismic data. Joint processing of the 2016 seismic and GPS data are underway. {\bf Broader Impacts}: Supported 2 postdocs, 2 graduate students. NSF Artists and Writers awardee Glenn McClure produced symphonic and choral works from the DRIS data. The Birch Aquarium at Scripps is developing an Antarctic exhibit featuring DRIS results. Publications: \cite{bromirski2015,diez2016,bromirski2017,chen2018,shen2018,chaput2018,white2019}. 
A project web site \href{https://scripps.ucsd.edu/centers/iceshelfvibes/}, suitable for informing the public, is maintained. 
Our research was featured on the {\urlcolor=cyan}
\href{https://www.youtube.com/watch?v=djesneud0Yg&fbclid=IwAR1zgpupmvZv2lFUy2ce2bLAgpIvi0M7OBW7P0koa0VwpwfMg5-8Pyg9hwE&app=desktop}{\bf The Colbert Report}!


% Cloniger
{\bf A. Cloninger}: ``A generalized framework for heterogeneous data fusion without point registration'', (NSF MSPRF, DMS-1402254, 07/2014-06/2017, \$150,000) This grant resulted in 12 accepted or submitted publications \cite{mishne2017diffusion,cloninger2018bigeometric,shaham2018provable,cloninger2017spectral, cloninger2017note, cloninger2017prediction,cloninger2016function, downing2017describing, bates2017outcome, katzman2018deepsurv,cheng2017two,cloninger2018people}.
%%%%
{\bf Intellectual merit:} \cite{mishne2017diffusion,cloninger2018bigeometric,shaham2018provable} study the algorithmic and theoretical aspects of deep learning in manifold contexts and for finding a encoding features that are shared across multiple networks or data sets.  \cite{cloninger2017spectral, cloninger2017note, cloninger2017prediction} study a related question of crafting new embeddings and latent spaces that move beyond the Laplace Beltrami eigenfunctions to embeddings involving predicting external functions, directed networks, and solutions to the wave equation.% all of which can have benefit in the context of aligning embeddings across data sets.  
\cite{cloninger2016function, downing2017describing, bates2017outcome, katzman2018deepsurv} focus on various applications of these frameworks in various medical problems.  And \cite{cheng2017two,cloninger2018people} developed methods for testing whether non-registered heterogeneous data sources were distributionally similar.
{\bf Broader impacts:}
The broader impact of the project was in generalizing the notion of heterogeneous data and avoiding heuristic approaches.  This created opportunities for interdepartmental collaborative efforts, continuing collaboration with the investigator's colleagues at the National Institutes of Health, Brigham and Women's Hospital, Cincinnati Children's Hospital Medical Center, and the Center for Outcome Research and Evaluation.  The grant also led to an undergraduate research project that has culminated in a well-cited paper, where the undergraduate was a lead author \cite{katzman2018deepsurv}.



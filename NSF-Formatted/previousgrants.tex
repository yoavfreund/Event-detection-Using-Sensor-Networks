
\section{results from previous grants}
\subsection{Sampling and quantization theorems for modern data acquisition}
%The investigator has not received prior NSF support.
%\vspace{5pt}\noindent{\bf Results from Prior NSF Support:} 
{Co-PI Saab: \em Sampling and quantization theorems for modern data acquisition (DMS-1517204, 08/01/2015--07/31/19,
\$160,404)} 
%   So far this grant has resulted in three journal articles \cite{KSW16, MS16, NSW16} and one journal submission \cite{SWY16}. \\
%   {\bf Intellectual Merit:} Two of the above works \cite{KSW16, SWY16} pertain to the quantization of compressed sensing measurements. Specifically, \cite{KSW16} is the first result showing that \emph{scalar} 1-bit quantization allows the magnitude of sparse signals to be recovered from compressed sensing measurements, provided an affine shift is employed. \cite{SWY16} shows that a combination of Sigma-Delta quantization and an encoding step based on a discrete Johnson-Lindenstrauss embedding allows for a near-optimal rate-distortion relationship. The other two papers quantify how using prior information (in the form of a possibly erroneous support estimate) in conjunction with weighted $\ell_1$ minimization allows for improved recovery guarantees from fewer compressed sensing measurements when the support estimate is accurate enough. 
% \\ {\bf Broader Impacts: }  In addition to dissemination of research results through seminars and workshop talks, Saab has developed and taught graduate courses on (1) compressed sensing and its applications and (2) applied and computational harmonic analysis. Additionally, he has mentored a (former) undergraduate (Steven Gagniere, currently at UCLA) in research on quantization of compressed sensing measurements (paper currently being written).   
This grant resulted in 9 published or accepted journal articles \cite{KSW16, MS16, NSW16, SWY16, NSW17, LS2018, IPSV2018, HS2018, FKS17}, and 4 conference papers \cite{NSW17_conf, FKS17_conf, IPSV17_conf, ILNS19}.  {\bf Intellectual Merit:}  \cite{KSW16, SWY16, FKS17, LS2018, ILNS19} study quantization of compressed sensing (CS) measurements,  dealing with  $1$-bit scalar quantization, compression of  bit-streams,  circulant measurement matrices, and low-rank matrices, and manifold-valued signals. \cite{HS2018} develops a state-of-the-art technique, and accompanying theory, for embedding datasets into the binary cube while preserving Euclidean distances.  \cite{NSW17} provides a framework for using $1$-bit  measurements for classification  and analyzes the case of two clusters. The papers \cite{MS16, NSW16} show that using prior support information and weighted $\ell_1$ minimization yield improved recovery  from fewer CS measurements. Finally, \cite{IPSV2018, IPSV17_conf} study phase retrieval from local measurements. The works use and develop tools in random matrix theory, mathematical signal processing, and applied harmonic analysis.  
  %Specifically, \cite{KSW16} is the first result showing that \emph{scalar} 1-bit quantization allows magnitude recovery from compressed sensing measurements, provided an affine shift is employed. \cite{SWY16} shows that Sigma-Delta quantization with an encoding step based on a discrete Johnson-Lindenstrauss embedding allows for a near-optimal rate-distortion relationship. \cite{FKS}The other two papers quantify how using prior information (in the form of a possibly erroneous support estimate) in conjunction with weighted $\ell_1$ minimization allows for improved recovery guarantees from fewer compressed sensing measurements when the support estimate is accurate enough. 
 {\bf Broader Impacts: }  In addition to dissemination of results through multiple seminars and workshop talks, Saab has developed and taught graduate courses on (1) compressed sensing and its applications, (2) applied and computational harmonic analysis, and (3) mathematical methods in data science. He has  mentored a UCSD undergraduate (S. Gagniere, currently at UCLA) in research on quantization of CS measurements, and UCSD graduate students (E. Lybrand, A. Nelson, B. Preskitt, J. Liang), and postdocs (A. Ma, T. Huynh),  some of whom are co-authors on the works above, while others are taking part in ongoing work. Two of the above mentioned graduate students and postdocs are female, one is an air-force officer, and at least two are members of under-represented groups.%\hfill\\

{\bf  P. Gerstoft}: 2014-2018; PLR-1246151 (SIO; 844k); Collaborative Research: Dynamic Response of the Ross Ice Shelf to Wave-induced Vibrations. Intellectual Merit: This project (DRIS) investigates the response of ice shelves to ocean wave forcing to (1) infer bulk elastic properties from signal propagation characteristics, and (2) to determine how IG wave and other gravity wave forcing propagates across the Ross Ice Shelf (RIS), and (3) to monitor seasonal variability of the RIS response and icequake activity. In 2014, 16 DRIS broadband stations were installed to acquire data for 2 years. Subsequently, 13 geodetic GPS stations were installed during the 2015 field season recovery of the first year’s seismic data. Joint processing of the 2016 seismic and GPS data are underway. {\bf Broader Impacts}: Supported 2 postdocs, 2 graduate students. NSF Artists and Writers awardee Glenn McClure produced symphonic and choral works from the DRIS data. The Birch Aquarium at Scripps is developing an Antarctic exhibit featuring DRIS results. Publications: [Bromirski et al., 2015; Diez et al., 2015; Bromirski et al., 2017; Chen et al., 2018]. A project web site [Bromirski, 2014], suitable for informing the public, is maintained. 



% Cloniger
{\bf A. Cloninger}:\alex{will shorten} NSF MSPRF, award number DMS-1402254, ``A generalized framework for heterogeneous data fusion without point registration'', from July 2014 to June 2017.  
%%%%
The original goals of this NSF fellowship were to use manifold learning techniques to generate a framework for data fusion from multiple modalities, even in situations where there is no point to point correspondence between the data structures.  The fellowship also had the goal of applications of such frameworks to medical imaging and cancer research.   

The intellectual merit was the aim of creating a parameter free representation of heterogeneous data.  A rigorous analysis of function extension to a common space lies on the border of statistical pattern learning and functional analysis, and is potentially transformative in both fields.  The project aimed to generalizes topological and geometric similarity to the level of data, allowing deep results in the study of sampling from manifolds to arise naturally.  And with a computational focus, this project allowed the end-user of the algorithm to analyze this fused data in a natural, easy to visualize space. 
The broader impact of the project was in generalizing the notion of heterogeneous data and avoiding heuristic approaches.  This data fusion framework is applicable in medical or geospacial imaging, characterizing psychological response surveys, or analyzing the pharmacodynamics of cancer drugs on a genomic level.  This created opportunities for interdepartmental collaborative efforts, continuing collaboration with the investigator's colleagues at the National Institutes of Health.  

This fellowship was very successful in a number of the outcomes originally set forth by the proposal, as well as bringing forward new questions and solutions that complement the proposal in ways that were unforeseen in its original formulation.  The main mathematical goal of the proposal, namely a framework for fusion of multiple modalities, was steadily addressed by answering a variety of specific questions.  We addressed the question of building high-dimensional distances between non-registered data sets by creating a novel anisotropic maximum mean discrepancy statistic that determines the regions of maximal difference, and is in the form of an optimization scheme that can be minimized in order to register two datasets together \cite{cheng2017two}.  We also used the same manifold learning framework to establish embeddings that are independent of the algorithmic parameters used to analyze the individual data set \cite{cloninger2016bigeometric,mishne2015diffusion}.  Finally, the work used a fast pairwise data adaptive earth movers distance to characterize the distance between a large number of non-registered datasets, specifically survey data \cite{cloninger2017people}.  
%The second goal of the original proposal, forming interdepartmental collaborations with the medical field, was also incredibly productive.  During the PIs time at Yale, he formed a successful collaborative group with the Center for Outcomes Research and Evaluation (CORE) in the Yale Medical School, which led to a number of significant results.  We examined the fusion of a large number of conflicting quality measurements of hospitals to generate a joint map that summarized distances between any two hospitals \cite{downing2017describing}.  We also used our fusion framework to create unsupervised clustering algorithms for determining whether a patient has one of several types of blood cancers, based off of an existing medical test known as flow cytometry.  This same approach can be used for a similar clustering in diffusion MRI of brains to determine inter-group similarities and differences \cite{cheng2017two}.  
The second goal of the original proposal, forming interdepartmental collaborations with the medical field, was also incredibly productive.  The PI formed successful and ongoing collaborations with researchers at the Center for Outcomes Research and Evaluation, as well as collaborators at NIH, Brigham and Women's Hospital, and Cincinnati Children's Hospital Medical Center \cite{cloninger2016bigeometric,downing2017describing,cloninger2017people}.  This also led to advising on additional important problems in the medical community, including individualized treatment effectiveness \cite{cloninger2016function}, and unsupervised clustering of flow cytometry measurements for blood cancers, and of diffusion MRI brain scans \cite{cheng2017two}. The grant also led to an undergraduate research project that has culminated in a well-cited paper, where the undergraduate was a lead author \cite{katzman2016deep}.

The opportunity to complete this fellowship at Yale led to several additional lines of research that were not outlined in the original proposal, but was equally important in career development. % Discussions and collaborations with CORE led to the development of a supervised manifold learning perspective for determining individualized treatment effectiveness of drugs \cite{cloninger2016function}.  
The chance to work with Professor Coifman and his research group has also led to studying the types of function classes that can be efficiently learned by deep neural networks \cite{shaham2016provable}, novel manifold learning kernels \cite{cloninger2017spectral}, and even advances in computation number theory \cite{cloninger2017suprema}.  
This award resulted in 12 accepted or submitted publications in a number of journals, including papers accepted in
Applied and Computational Harmonic Analysis,
Proceedings of the AMS,
Journal of Magnetic Resonance,
PloS One,
SPIE Wavelets and Sparsity XVII, 
and SampTA Proceedings.
\iffalse
This award also provided the ability to travel to a number of conferees to disseminate the information, including
SampTA 2015,
SPIE 2017,
workshop at Banff International Research Station 2015,
workshop at Casa Mathematica Oaxaca 2016,
a talk at Institute for Mathematics and Applications 2016,
and a talk at The Chicago Chapter of the American Statistical Association 2017.
\fi


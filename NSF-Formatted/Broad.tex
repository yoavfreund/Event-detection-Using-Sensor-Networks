\section{Broader Impacts}
\subsection*{Societal benefits: practical applications}
Sensor networks are an important emerging technology with applications
in retail, manufacturing, security and medicine. These networks
collect vast amounts of raw data, most of which is not relevant to the
task of the system as a whole. On the other hand, such systems face strict constraints on the power and bandwidth available to each sensor.
Sensors are expected to operate for years on a small battery or by
foraging energy from the environment. A related problem is the
bandwidth, range and energy consumption of wireless communication
protocols, which greatly limit the data-rate sent in an out of each
sensor. Our proposed work
will enable a new generation of sensor networks which will impact all
aspect of modern society.
\subsection*{Cross disciplinary collaboration}
The PIs all have a history of cross disciplinary collaboration, not only across the HDR TRIPODS disciplines but also with neuro-scientists, geoscientists, medical doctors, \rayan{All: insert your stuff here please}. NSF support for this project will allow the PIs to develop new cross-disciplinary collaborations, particularly with researchers in the areas of embedded computers, robotics, and medical sensing. Indeed, practitioners from these areas are likeliest to adapt and apply the results of the proposed research. So, we plan to actively foster collaborations with them, in anticipation of joining forces for phase II of HDR TRIPODS. To that end, we will greatly benefit from the new Halicioglu Data Science Institute (HDSI) at UCSD, which boasts several world class researchers (\rayan{Yoav, Peter, Piya, any particular names we should throw in here?} in those areas among its affiliated members. 
\subsection*{Teaching and mentorship}
{\bf Mentorship} In addition to the proposed scientific content, a main goal of this proposal is to recruit and train highly qualified personnel for the workforce of tomorrow. Exploiting the fact that our team members are all UCSD faculty with affiliations to HDSI, we will recruit and advise graduate and post-graduate students as a
group. We will actively recruit students who have a strong
mathematical background and are able and willing to implement
algorithms as reusable code. \emph{Equally importantly, we will actively seek out students from underrepresented groups in STEM.}

{\bf Teaching} In addition,  HDSI which is a new institute at UCSD, houses an undergraduate program in data science with both major and minor degrees awarded.  It is also encouraging the development of graduate classes focusing on data science, in several departments (including Computer Science, Electrical Engineering, and Mathematics). The PIs will thus develop undergraduate and graduate courses on mathematical and computational aspects of data science, with the graduate ones based in part on several topics courses
they have already taught. This will help ease the entry of UCSD  students into the field. The
results of the proposed research will be incorporated into these courses. 

{\bf Mentorship in collaboration with industry} HDSI is piloting a program whereby undergraduates pursuing data science degrees work on projects proposed and partially funded by industry, under the supervision of an HDSI faculty member. With the help of HDSI, we will seek out such collaborators in industry, in areas related to the work proposed herein and use the opportunity for undergraduate mentorship. Here again we will recruit strong undergraduates with a particular focus on underrepresented groups in STEM.

\subsection*{Disseminating knowledge}
Once again capitalizing on the unique opportunities and resources provided to us by HDSI, we will (in addition to the usual conference and workshop participation) ourselves organize a workshop at UCSD for leading data scientists from across the HDR TRIPODS disciplines. In addition to presenting their (and our) work, the workshop will actively encourage collaborations among its participants by hosting several open-problem sessions.

\section{Timeline of Activities}

Our aim is to develop an interdisciplinary group, to that end, we will immediately and actively recruit students and a postdoctoral scholar who are strong in both mathematics and programming. Within the first three months we will curate data-sets and define specific challenges for the analysis of these data-sets. Each graduate student will be assigned at least one challenge and will start working on it upon joining the group. Challenges will be made open, Kaggle style, to encourage student in our data science programs and elsewhere to compete for the best performance.

In addition to the generation of paper publications, students will be expected to generate github repositories containing code and explanations of their work on the challenges. The notebooks and challenge results will be evaluated continuously by the student's advisor and be the whole group. It is expected that this procedure will generate higher quality research output.

The investigators, along with the postdoc, will immediately begin working on the theoretical foundations of the project while simultaneously training the graduate students. 

Scientifically, we anticipate making steady progress, essentially tackling the problems in each of the sections in parallel, as the update on each topic will influence the other. 

In anticipation of participating in the formation of Phase II Institute, beginning in year 2, we will engage scientists and engineers from application domains including embedded systems, robotics, and medical sensing in collaborative research. We will work with those collaborators to create datasets and challenges specific to each application area.
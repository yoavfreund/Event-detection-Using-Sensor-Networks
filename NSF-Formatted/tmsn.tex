\section{Tell me something new (Yoav)}
We propose a general framework for designing and analyzing sensor networks. Suppose we have two sensors two sensors whose task is to track state of target whose trajectory is defined by $\state(t)$. Suppose the sensors use a dynamic model to predict future states. For the sake of concreteness, suppose a Newtonian model of motion is used which predicts constant velocity of the target unless a disruptive force acts on it.

Let $\estate_1(t),\estate_2(t)$ be the estimates of the target location for each sensor. {\em In addition, each sensor maintains an estimate of the estimate of the other sensor.} $\estate_{1,2}(t),\estate_{2,1}$. Each sensor updates it's estimate of the velocity of the location of the target according to the signal it measures, but it does not update it's estimate of the other's estimate.  If the two estimate are close to each other $\estate_i(t) \approx \estate_{i,j}(t)$ then sensor $i$ sends no information to sensor $j$. On the other hand, if $\estate_i(t)$ is far from $\estate_{i,j}(t)$, then $\estate_{i}$ is transmitted from sensor 
$i$ to sensor $j$. Thus if the target is moving in constant speed, uninterrupted, there is not communication between the sensors.

The basic idea here is that a sensor sends out information only if
that information cannot be predicted by the receiver. Similar ideas
have been used in arithmetic coding and \yoav{I think} in
$\Sigma\Delta$ encoding.

Recently, PI Freund~\cite{TMSN} proposed an asynchronous computation
model called ``Tell Me Something New'' in which each agent broadcasts
a message only when the estimate it computes differs from the existing
estimates in a statistically significant way.

 One important application of sensor networks is to monitor activity
 and identify anomalies. Examples include: building security systems,
 factory floors, highway monitoring, health monitoring for the sick or
 elderly and many others.

 On its face, this might seem like an under-constrained impossible
 problem. However, note that for all of the environments listed above
 there is a highly repetitive pattern from day to day and from week to
 week. Add to that the sensors are stationary, and one would expect
 that most sensors observe highly regular and highly predictable
 patterns.
 
 The approach we propose in this case is that each sensor creates a
 model of the characteristics of the signals that it observes during
 normal operations. It alerts neighboring sensors if it observes
 something that is abnormal, i.e. a signal that has very low
 probability according to the model. When several sensors send an
 alert with a short time window, and when the alerts are consistent
 with each other, a global alert is sent to the human operators.


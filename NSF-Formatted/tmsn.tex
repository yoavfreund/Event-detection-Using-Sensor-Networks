\section{Tell Me Something New}\label{sec:TMSN}

Consider a video system for monitoring a highway. Lets say that the
highway is 100 miles long and that we have a 5 cameras per mile,
i.e. 500 cameras. Each camera generates a large raw data stream, on
the order of 1Gbit per second. Using video compression this stream can
be brought down to 1-10Mbit/sec. Totalling .5-5Gbit/sec. A common approach
is to send all of this data to a centralized site where a team of
analysts monitor it to identify accidents, road hazards etc.

Several issues should be noted:
\begin{itemize}
\item A high bandwidth channel is needed to transmit the information
  to the center. If this channel is wireless, this transmission will
  be costly in terms of energy and available bandwidth.
\item Manually monitoring 500 video channels is a very labor intensive task. 
\item Highway traffic is, for the most part, highly
  predictable. Analyst are, for the most part, have no interest in normal traffic. 
  They need to be alerted to {\em abnormalities} such as an accident, a hazardous 
  object on the road or a speeding car.
\end{itemize}

We propose a general framework for designing sensor networks for
anomaly detection called ``tell me something new''~\cite{TMSN}. In
this approach, sensors analyze the data they collect, and send message
to the center only when it estimates that this information is
significantly different than what the center will predict without the
information.

We assume that the cameras are placed at locations and orientations that are fixed for years.
Each camera learns, over time, a predictive model of the traffic on the small segment of 
highway that it observes. Inputs to this model might include time of day, day of week, and day of year, visibility, snow, rain etc. Outputs might include the distributions of car speed in each lane, and number of cars per minute.
The central location also has the predictive model, which is updated from time to time to reflect long term changes. 

In normal operation the camera compares it's observations to the model. If traffic behaviour is as predicted, the camera sends no information to the center. On the other hand, unpredicted behaviour such as a speeding car or an accident will cause the camera to send information to the center, potentially with additional high resolution camera footage.

Estimating whether the distribution of a sequential signal normal is usually based on a "sliding window" technique. Such techniques require choosing the size of the window, which in turn depends on the distribution of the signals.
One way out of this infinite regress is to use a statistical technique developed by Abraham Wald~\cite{wald1973} 
called {\em sequential analysis}. In~\cite{TMSN} we demonstrate how to use sequential analysis in this way.

TMSN can be used in the learning process itself. In particular, if the sensors are using boosting~\cite{boostingbook} then TMSN allows for a new distributed boosting approach. In this approach~\cite{fastboost} different sensor/learners search for 
improvements in the form of weak rule. If a rule is found that is significantly better than random guessing, it is broadcast to the other sensors. The other sensors test the new rule on their own data and, if they observe a significant advantage, add the weak rule to their own strong rule. This defines a robust asynchronous distributed learning which is currently under development.
\documentclass{article}
\usepackage[utf8]{inputenc}
\usepackage{amsmath}
\usepackage{amssymb}
\usepackage{fullpage}
\usepackage{color}

\usepackage{textcomp}
\usepackage{ifthen}
\usepackage{listings}
\usepackage{fancyvrb}
\usepackage{verbatim}
%\usepackage[noend]{algorithm2e}

%\usepackage[pdftex]{graphicx}
\usepackage{graphicx}
\usepackage{subfig}
\usepackage{calc}
\usepackage{float}

\usepackage{url}

% For generating custom lists (similar to table of contents, list of figures
% etc.)
\usepackage[subfigure]{tocloft}

\usepackage{hyperref}
\usepackage[all]{hypcap}
\usepackage{latexsym}

\usepackage{enumitem}

% Remove spacing around section headings
\usepackage[compact]{titlesec}
\titlespacing{\section}{0pt}{*0}{*0}
\titlespacing{\subsection}{0pt}{*0}{*0}
\titlespacing{\subsubsection}{0pt}{*0}{*0}

%\usepackage[left=1.05in,right=1.05in,top=1in,bottom=1in]{geometry}

% tweak or comment out the baselinestretch to control how much space for hand-writing comments you get between the lines
%\renewcommand{\baselinestretch}{1.8}

% Compress whitespace around itemize and enumerate
\setitemize{topsep=.0em,parsep=0pt,partopsep=0pt,labelsep=0.2em,itemsep=0pt,leftmargin=0.5em}
\setenumerate{topsep=.0em,parsep=0pt,partopsep=0pt,labelsep=0.2em,itemsep=0pt,leftmargin=0.5em}

% Remove paragraph indentation
\setlength{\parindent}{0pt}

% Remove paragraph spacing
\setlength{\parskip}{0pt}
\setlength{\parsep}{0pt}
\setlength{\headsep}{0pt}
\setlength{\topskip}{0pt}
\setlength{\topmargin}{0pt}
\setlength{\topsep}{0pt}
\setlength{\partopsep}{0pt}

% Decrease the line spread slightly
\linespread{1.02}


\DefineVerbatimEnvironment%
{tree}{Verbatim}{frame=single,fontsize=\scriptsize}

\DefineVerbatimEnvironment%
{treeNumber}{Verbatim}{frame=single,numbers=left,fontsize=\scriptsize}
\title{Project Summary} 

\newtheorem{definition}{Definition}

%\renewcommand{\comment}[3]{}  % suppress comments
\renewcommand{\comment}[3]{{\color{#1} {\bf #2 :} #3}}

\newcommand{\yoav}[1]{\comment{magenta}{Yoav}{#1}}
\newcommand{\piya}[1]{\comment{blue}{Piya}{#1}}
\newcommand{\peter}[1]{\comment{pink}{Peter}{#1}}
\newcommand{\rayan}[1]{\comment{red}{Rayan}{#1}}
\newcommand{\alex}[1]{\comment{green}{Alex}{#1}}

%math symbols
\newcommand{\R}{\mathbb{R}}
\newcommand{\rad}{\mathcal{R}}
\newcommand{\sign}{\mathrm{sign}}

\newcommand{\state}{\theta}
\newcommand{\estate}{\hat{\theta}}
\newcommand{\Vstate}{\Theta}
\newcommand{\eVstate}{\hat{\Theta}}
\newcommand{\Sstate}{\psi}
\newcommand{\eSstate}{\hat{\psi}}
\newcommand{\VSstate}{\Psi}
\newcommand{\eVSstate}{\hat{\Psi}}
\newcommand{\env}{\mathbf{E}}
\newcommand{\transfer}{\mathbf{\Phi}}

\newcommand{\signal}{\mathbf{x}}
\newcommand{\Vsignal}{\mathbf{X}}
\newcommand{\dsignal}{\mathbf{y}}
\newcommand{\Vdsignal}{\mathbf{Y}}
\newcommand{\dtime}{t}
\newcommand{\ctime}{\tau}
\newcommand{\y}{\mathbf{y}}
\newcommand{\q}{\mathbf{q}}
\newcommand{\x}{\mathbf{x}}
\newcommand{\uu}{\mathbf{u}}

\begin{document}
\section{Overview}
The fact that humans have {\em two} eyes and {\em two} ears has clear
benefits. Our two eyes provide us with depth perception. Our two ears
allow us to detect the direction from which a sound is coming.

The same holds for artificial sensors. Multiple sensors with
overlapping receptive fields can produce better representations of the
environment. However, to produce such representations, the data
streams generated by the sensors need to be compared and combined. One
approach is to send all of the signal to a center and use a single
computer to combined them. This approach requires high bandwidth
communication and powerful computers and does not scale well to
networks with hundreds or thousands of sensors.

An alternative is {\em pushing computation to the edge}. Instead of
performing all of the computation on a central machine, each sensor
has some computing power which it uses to perform some of the
computation and reduce the amount of information that needs to be
sent. This reduction is not compression as we are not asking for a
reconstruction of the original signal. Instead what we want to produce
is a high-level representation of the environment. For example, the
location of an object, or a count of the people in a particular
area. Computing this high level representation will usually require
much less information from each sensors than a compressed version of
the raw signal.

We propose a novel approach we call {\em Learning Sensor Networks}.
This approach assumes that sensors are placed at fixed locations and
remain there for a significant length of time.  Each sensor uses {\em
  statistical learning} to model it's environment. It communicates
with neighboring sensors in order to know their models and to
calibrate parameters such as relative location. After an initial
training period, which might last minutes, days or months depending on
the context, the network can distinguish normal from abnormal
behaviour. Based on that ability, it will transmit to the other
sensors only a {\em sketch}~\cite{} which captures the novel, or
salient aspect of the signals.

Learning sensor networks promise much lower energy and bandwidth
consumption than current sensor networks. To realize this potential we
plan to develop new new mathematical models, signal processing methods
and distributed algorithms for computing and combining sketches.

\section{Intellectual Merit}

\section{Broader Impacts}
\subsection*{Societal benefits: practical applications}
Sensor networks are an important emerging technology with applications
in retail, manufacturing, security and medicine. These networks
collect vast amounts of raw data, most of which is not relevant to the
task of the system as a whole. On the other hand, such systems face strict constraints on the power and bandwidth available to each sensor.
Sensors are expected to operate for years on a small battery or by
foraging energy from the environment. A related problem is the
bandwidth, range and energy consumption of wireless communication
protocols, which greatly limit the data-rate sent in an out of each
sensor. Our proposed work
will enable a new generation of sensor networks which will impact all
aspect of modern society.
\subsection*{Cross disciplinary collaboration}
The PIs all have a history of cross disciplinary collaboration, not only across the HDR TRIPODS disciplines but also with neuro-scientists, geoscientists, medical doctors, \rayan{All: insert your stuff here please}. NSF support for this project will allow the PIs to develop new cross-disciplinary collaborations, particularly with researchers in the areas of embedded computers, robotics, and medical sensing. Indeed, practitioners from these areas are likeliest to adapt and apply the results of the proposed research. So, we plan to actively foster collaborations with them, in anticipation of joining forces for phase II of HDR TRIPODS. To that end, we will greatly benefit from the new Halicioglu Data Science Institute (HDSI) at UCSD, which boasts several world class researchers (\rayan{Yoav, Peter, Piya, any particular names we should throw in here?} in those areas among its affiliated members. 
\subsection*{Teaching and mentorship}
{\bf Mentorship} In addition to the proposed scientific content, a main goal of this proposal is to recruit and train highly qualified personnel for the workforce of tomorrow. Exploiting the fact that our team members are all UCSD faculty with affiliations to HDSI, we will recruit and advise graduate and post-graduate students as a
group. We will actively recruit students who have a strong
mathematical background and are able and willing to implement
algorithms as reusable code. \emph{Equally importantly, we will actively seek out students from underrepresented groups in STEM.}

{\bf Teaching} In addition,  HDSI which is a new institute at UCSD, houses an undergraduate program in data science with both major and minor degrees awarded.  It is also encouraging the development of graduate classes focusing on data science, in several departments (including Computer Science, Electrical Engineering, and Mathematics). The PIs will thus develop undergraduate and graduate courses on mathematical and computational aspects of data science, with the graduate ones based in part on several topics courses
they have already taught. This will help ease the entry of UCSD  students into the field. The
results of the proposed research will be incorporated into these courses. 

{\bf Mentorship in collaboration with industry} HDSI is piloting a program whereby undergraduates pursuing data science degrees work on projects proposed and partially funded by industry, under the supervision of an HDSI faculty member. With the help of HDSI, we will seek out such collaborators in industry, in areas related to the work proposed herein and use the opportunity for undergraduate mentorship. Here again we will recruit strong undergraduates with a particular focus on underrepresented groups in STEM.

\subsection*{Disseminating knowledge}
Once again capitalizing on the unique opportunities and resources provided to us by HDSI, we will (in addition to the usual conference and workshop participation) ourselves organize a workshop at UCSD for leading data scientists from across the HDR TRIPODS disciplines. In addition to presenting their (and our) work, the workshop will actively encourage collaborations among its participants by hosting several open-problem sessions.

\end{document}

% Collaboration and Evaluation Plan as a separate Supplementary Document
% (limited to 5 pages). This plan must describe the expertise in the
% relevant disciplines provided by the PIs as required above under "Who
% May Serve as PI" as well as plans for working together to meet the
% goals of the program. The Collaboration and Evaluation Plan must also
% describe clear measures of success for the project, including
% developing capability and capacity for a potential Phase II, and a
% plan for evaluating success. Proposals without this document will be
% returned without review.

